\begin{abstract}

With enough power, any problem becomes tractable.
Since the inception of wireless sensor networks, researchers have continuously searched for ways to do more with less.
%Indirectly, this is really a search for more energy, or ways to use currently available energy more efficiently.
Integrated circuits and sensors have continued to shrink in size, cost, and active and quiescent power.
This has resulted in sensors with increasing computational power and longer lifetimes.
However, the options for and quantity of power available for a wireless sensor has comparatively stagnated.
The energy density of non-rechargeable batteries as well as photovoltaic efficiency have approached a plateau.
This results in wireless sensors that are either constrained in lifetime, or constrained in power.

Technology improvements like 
%Researchers have capitalized on new technologies, like 
integrated circuits that perform efficient maximum power tracking and voltage boosting, near-threshold computing, the advent of non-volatile memory technologies, and the rapid improvement of supercapacitor technology has 
%the advent of new non-volatile memory, or the improvement of supercapacitor technology 
enabled the development of sensors that can operate entirely on harvested power without batteries, and without a finite lifetime. 
However, the lack of a reliable power source results in a system that is fundamentally unreliable.
%Proponents of batteryless systems believe batteries are a threat to the future of wireless sensing, and leaving them behind is the only path forward~\cite{hester2017future}. 
Despite this significant flaw, researchers have pursued this design archetype tirelessly, producing an impressive corpus of methods, systems, and solutions that attempt to improve batteryless design.
Proponents of batteryless systems are convinced that batteries are a threat to the future of wireless sensing~\cite{hester2017future}, and that batteryless sensing is the only way forward.
Despite this, batteryless systems have not seen widespread adoption by industry. 
There is a rift of design understanding between those who value reliability, and those who do not.

%This dogmatic pursuit of batteryless design has resulted in an artificial shrinking of what is actually a multi-dimensional design space.
%By abandoning valid design options, 
%the development of batteryless systems has resulted in a rift of design
%The wireless sensor community has lost track of how to design devices to meet the demands of real world applications.

The core argument of this dissertation is that there is not a single design dogma, be it batteryless or battery-powered, that 
can provide a solution for \textit{all} applications.
Instead, the correct design process resembles a balancing act of the inclusion and sizing of energy harvesting, rechargeable, and non-rechargable energy storage.
Depending on application requirements, this balance must change to address them.
This dissertation explores the wireless sensor application and design space, identifying the appropriateness of different approaches qualitatively and quantitatively. This dissertation details a deep dive on the system-level effects of harvester size and rechargeable and non-rechargeable energy capacity on wireless sensor application performance.
This dissertation provides a quantitative comparison between energy storage options and reevaluates the many qualitative claims made against batteries by batteryless proponents, concluding that many of them are without merit. 
As a final contribution, this dissertation looks ahead at new more complex sensing modalities and the limits of current embedded computation, within the context of energy harvesting. 

\end{abstract}

%%% Local Variables:
%%% mode: latex
%%% TeX-master: "thesis"
%%% End:
