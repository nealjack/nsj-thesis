\begin{abstract}

Any problem becomes tractable with enough power.
Since the inception of wireless sensor networks, researchers have searched for ways to do more with less.
%Indirectly, this is really a search for more energy, or ways to use currently available energy more efficiently.
Integrated circuits and sensors have continued to shrink in size, cost, and active and quiescent power.
This has resulted in sensors with increasing computational power and longer lifetimes.
By comparison, however, the options for and quantity of power available for a wireless sensor has stagnated.
The energy density of non-rechargeable batteries as well as photovoltaic efficiency have approached a plateau.
As a result, wireless sensors are constrained in either lifetime or power.

Technology improvements like 
integrated circuits that perform efficient maximum power tracking and voltage boosting, near-threshold computing, the advent of non-volatile memory technologies, and the rapid improvement of supercapacitor technology has 
enabled the development of sensors that can operate entirely on harvested power without batteries, and without a finite lifetime. 
But the lack of a reliable power source necessarily results in a system that is fundamentally unreliable.
Despite this significant flaw, researchers have pursued this design archetype tirelessly, producing an impressive corpus of methods, systems, and solutions that attempt to improve batteryless design.
Proponents of batteryless systems are convinced that batteries are a threat to the future of wireless sensing, and that batteryless sensing is the only way forward.
Despite this, batteryless sensing has not seen widespread adoption by industry. 
There is a rift of design understanding between those who value reliability, and those who do not.

%This dogmatic pursuit of batteryless design has resulted in an artificial shrinking of what is actually a multi-dimensional design space.
%By abandoning valid design options, 
%the development of batteryless systems has resulted in a rift of design
%The wireless sensor community has lost track of how to design devices to meet the demands of real world applications.

The core argument of this dissertation is that there is not a single design dogma, be it batteryless or battery-powered, that 
can provide a solution for \textit{all} applications.
Instead, the correct design process must involve a balancing act of the inclusion and sizing of energy harvesting, rechargeable, and non-rechargable energy storage to meet the goals of the application.
This design space is large and difficult to navigate, resulting in many system designers defaulting to following a predetermined design template archetype instead of fully reasoning about their application and its requirements.
In this dissertation, 
we develop a design framework for energy harvesting systems that provides reasoned guidance for the inclusion and sizing of various power supply elements. In particular, we develop analytical and simulation tools to size rechargeable energy capacity in a more reasoned way than current heuristics and arbitrary methods.

To develop this design framework, this dissertation
explores previous wireless sensor applications, identifying the appropriateness of different approaches qualitatively and quantitatively. 
We explore the system-level effects of harvester size and rechargeable and non-rechargeable energy capacity on wireless sensor application performance.
To determine rechargeable energy capacity selection and sizing, we develop a novel heuristic for determining the minimum sufficient capacity for a sensor workload and expected energy income.
We verify this heuristic through the use of a custom wireless sensor energy state simulator to estimate energy utilization and system performance. 
To identify technology options for energy capacity, we quantitatively compare energy buffer types and reevaluate the many qualitative claims made against rechargeable batteries by batteryless proponents, concluding that many of them are without merit. 
Finally, we utilize the design framework developed within this dissertation, including the heuristics and simulation tool, to design and implement wireless sensor systems to address two real indoor sensing applications that achieve long-lived operation with consistent and reliable sensing. 

\end{abstract}

%%% Local Variables:
%%% mode: latex
%%% TeX-master: "thesis"
%%% End:
