\section{Conclusions}
\label{sec:conc}

%This idea was predicated on the
%assumption that capacitors were, by many metrics,
%the best energy storage option available, despite their small capacity.
%claims made by the
%sensor networking community:
%we find many of these
%assumptions are not applicable when considering newer technology
%designing sensors for the common case,
%and that
%we find that the gains provided by increased capacity are numerous.
%Secondary cells can
%provide larger energy buffers,
%Recent technology developments point to rechargeable batteries with sufficient capacity that are small, cheap, and offer impressive lifetimes.
%are
%long-lived, adaptive, and capable rather than immortal, single-purpose but hi
%
%Future sensor designers can capitalize on these battery improvements and the
%capacity they enable to build reliable, long-lived, and capable sensor deployments,
%allowing complex and interesting applications.
%In retrospect, we see a large body of
%work that is
%devoted to making intermittency less painful by proposing complex
%software and hardware solutions, when instead intermittency could have been
%avoided altogether for most applications by just using batteries.

%Instead, many of the headaches of
%intermittency can be completely avoided by just using batteries. Going forward
%as a community, we need to make sure to quantitatively evaluate our assumptions
%and clearly define and evaluate intended use cases.

Our results, along with recent advances in battery technology, suggest that we
are now able to build energy harvesting sensors with greater energy storage
capacity. These sensors will not need to micro-manage their energy state and
can instead adapt their lifetime and energy usage over
the course of weeks, months, or not at all. This push toward
reliable and capable sensors will enable more standard programming models,
energy management techniques, and ultimately useful applications to be built on
top of dense and ubiquitous energy harvesting sensor deployments.
