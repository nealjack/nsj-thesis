% ABSTRACT

%Battery-powered sensors have long been the standard for simple and
%reliable sensor deployments, however as the power of individual components
%lowers, energy harvesting becomes an ever more viable means of powering
%sensors.
%even in deployment scenarios with low amounts of ambient energy.
%
%Today most energy harvesting systems rely entirely on harvested energy,
Today, most sensors that harvest energy
%attempting to subsist on the small amounts
%of energy harvested
from indoor solar, ambient RF, or thermal gradients
%rely entirely on this harvested energy,
buffer small amounts of energy in capacitors as they intermittently work through a sensing task.
%
While the utilization of capacitors for energy storage affords these systems
indefinite lifetimes, their low energy capacity
necessitates complex intermittent programming models for state retention and
energy management.
%
However, recent advances in battery technology lead us to reevaluate the impact
that increased energy storage capacity may have on the necessity of these
programming models and the reliability of energy harvesting sensors.

In this paper, we propose a capacity-based framework to help structure
energy harvesting sensor design,
analyze the impact of capacity on key reliability metrics
using a data-driven simulation, and
consider how backup energy storage alters the design space.
%
We find that for many designs that utilize solar energy harvesting, increasing
energy storage capacity to 1-10~mWh can obviate the need for intermittent
programming techniques, augment the total harvested energy by 1.4-2.3x, and
improve the availability of a sensor by 1.3-2.6x.  We also show that a hybrid
design using energy harvesting with a secondary-cell battery and a backup
primary-cell battery can achieve 2-4x the lifetime of primary-cell only designs
while eliminating the failure modes present in energy harvesting systems.
%
%By modeling the charge-discharge patterns of energy stores in indoor solar
%harvesting environments, we
%find that increasing energy storage capacity
%afforded by rechargeable batteries
%allows a sensor to capture 1.5-2.2x the available energy and
%increases its availability
%by a factor of up to 2.5x.
%
%We also consider the addition of a primary-cell battery
%as a backup energy store in addition to energy harvesting and find that this
%hybrid design can achieve 4-6x the lifetime of primary-cell only designs while
%eliminating the failure modes present in energy harvesting systems.
%
Finally, we implement an indoor, solar energy harvesting sensor based on
our analysis and find that its behavior aligns with
our simulation's predictions.



%rely entirely
%on harvested energy, buffering small amounts in
%this new class of sensors relied entirely on harvested energy, buffering
%small amounts in capacitors as they intermittently worked through a sensing task.

%they suffer from short lifetimes and
%high maintenance costs associated with battery replacement.
%In response, a new class of sensors emerged that eschew batteries and rely entirely on harvested
%energy, buffering small amounts in a capacitor to intermittently work
%through a sensing task.
%While these systems
%achieve a power supply with an indefinite lifetime,
%avoid batteries,
%which are dismissed as expensive, failure-prone,
%fragile, and temperature sensitive,
%they must tolerate the decreased
%availability, lower energy utilization,
%and more complex programming models inherent to a low-capacity,
%intermittent design.
%This work reevaluates the use of batteries and argues for sensor designs that utilize rechargeable and non-rechargeable batteries in tandem.
%, both rechargeable and non-rechargable,
%bridges the gap between these two design points,
%advocating
%for their use in energy harvesting sensors
%to extend the lifetime of
%a sensor node,
%observing that the increased energy capacity afforded by rechargeable batteries can increase harvestable energy utilization and in turn, system availability, responsiveness, and capability.
%Additionally, the inclusion of non-rechargeable batteries
%grants a sensor a finite non-intermittent lifetime.
%By modeling the charge-discharge patterns of energy stores in indoor solar
%harvesting environments, under both periodic and event-driven workloads, we
%observe that the increased energy capacity afforded by rechargeable batteries
%can capture 1.5-2.2x the available energy of capacitor-only designs.
%For the majority of workloads and harvesting conditions considered, this
%increase in available energy
%%substantially
%%increases system availability,
%%responsiveness, and capability,
%allows a modeled sensor to perform 100\% of expected tasks on time, compared to 40-80\% for a system that utilizes capacitors.
%The inclusion of a non-rechargeable battery
%achieves 4-6x the lifetime of primary-cell only designs. We consider new
%battery technology and management techniques, and conclude that many of the
%past criticisms against their use are outdated or mitigated for many
%applications in human-centric environments.
%%We conclude that power supplies that utilize batteries
%%are suitable for many sensor applications in human-centric environments,
%%ensure a minimum 100\% available and responsive lifetime and entirely avoid intermittency throughout the sensor's lifetime.
%For these applications, systems
%that employ batteries can entirely avoid the reduced availability, complex
%energy management, and special-purpose hardware platforms associated with
%intermittency and still achieve a long, but finite, lifetime.

%We support these claims by modeling the charge-discharge patterns of energy
%stores placed in real energy harvesting scenarios under periodic and event
%based workloads, and we defend our use of batteries by showing that
%the claims against them are either unfounded, outdated, or easily mitigated
%for the vast majority of real use-cases.
%
%
%
%By modeling the charge-discharge patterns of energy stores placed in real
%energy-harvesting scenarios under periodic and event-based workloads, we show
%the performance increases associated
%
%of
%the sensor node. To support these decisions we model the charge-discharge patterns
%of energy stores when placed in real, energy-harvesting scenarios under periodic
%and event-based workloads,
%
%
%
%By forgoing the high capacity of
%batteries however, these everyla
%
%
%however they inherently trade-off
%reliability
%
%Harvesting ambient energy to power distributed sensor nodes has the
%potential to extend their lifetimes and increase their performance compared
%to non-harvesting, battery-powered designs. Sensor platforms presented
%in prior work take advantage of this harvested energy by temporarily storing it in
%capacitors before using it to perform a small number of operations, intermittently
%working through a sensing task. Some authors
%
%Energy harvesting has the potential to extend the lifetime and increase
%the available
%
%however current
%energy harvesting architectures
%
%Energy harvesting has the potential to power distributed sensing nodes,
%extending their lifetimes, increasing the energy available to
%perform sensing tasks, and reducing the maintenance cost of a sensor
%deployment. In the past five years a trend has even
%emerged pushing for ``perpetual'' batteryless sensors,
%supported by claims that batteries are fragile, toxic, temperature
%sensitive, and prone to long-term cycling failure. Alternative
%designs temporarily store harvested energy in capacitors before performing
%a small number of operations, intermittently working through a sensing task.
%These intermittent designs, however, are inherently prone to low reliability
%for many workloads, present users with non-standard programming models to mask
%their intermittency, and make poor utilization of the harvestable ambient
%energy.
%
%We argue that utilizing batteries, both rechargeable and non-rechargeable cells,
%can solve these problems for the vast majority
%of real workloads and deployment scenarios while still providing
%the key benefits of energy harvesting.
%By analyzing the reliability, lifetime and harvestable energy utilization
%for a variety of energy storage systems we motivate the need to use
%batteries, and we welcome their use by
%showing that many claims against them are unfounded, outdated or easily mitigated.
%To perform the analysis, we model the charge-discharge patterns of
%energy stores when placed in real, energy-harvesting
%scenarios under periodic and event-based workloads, and we evaluate this
%model by comparing it to implementations of several points in the design space.
%We show that for reasonable energy storage sizes, workloads, and
%energy availability, sensor
%nodes utilizing batteries along with energy harvesting can capture \hl{6x} the available
%energy of capacitor-only designs and achieve at least \hl{4x} the lifetime of
%primary-cell only designs, all while avoiding the poor reliability and painful
%programming models associated with intermittency.
