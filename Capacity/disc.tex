\section{Discussion}
\label{sec:disc}
While we have verified that our model generally matches reality, many
assumptions were made that impact its predictions. Specifically,
lifetime estimates produced by the model can be described as optimistic. We are
skeptical of the reality of these lifetimes when considering the myriad of
factors not considered by our model.\\

\vspace{-6pt}
\noindent
\textbf{Model Limitations.}
Many assumptions were made when designing the model and considering different
parameters and behaviors.  In particular, many dynamic parameters are modeled
as static. For example, leakage is dependent on state of charge and age,
but we assume a static leakage rate. Similarly, many
components like solar panels and boost regulators have dynamic efficiencies,
that we consider to be static. Nominal voltages for solar panels and
batteries are used
%throughout
to calculate energy output and capacity,
%in our model,
but they
are really dynamic. In addition, we do not consider the effects of temperature
and ignore cycle capacity degradation for supercapacitors and
secondary-cells. We also assume that every platform can wake up
and perform a task at any point in its voltage curve. In reality,
many intermittent systems must recharge to a threshold voltage. We attempt
to ensure that these assumptions favor intermittent systems over
larger capacity designs so that
our conclusions are not distorted by inaccuracies in the model.\\

\vspace{-6pt}
\noindent
\textbf{Theoretically Infinite vs. Reality.}
We discuss capacitor-based systems as perpetual throughout the
paper and even predict that \name will have 30-50 years lifetimes
under some configurations and harvesting conditions. While in theory
this is true, it is highly unlikely that any sensor nodes we build and deploy today
will be functional, let alone relevant in 50\,years.
We expect physical degradation of not only energy storage components, but
also the silicon itself. The MCU architectures in use on these systems have
only been commercially available for 1-2 decades. This makes it
difficult to state that failure rates will remain low for over five.
Components like MEMS sensors and oscillators
will age and lose calibration well before this time.
Outside of physical degradation, we must also consider the potential decline of common
protocols and standards that dictate wireless and security functions, and special
care will need to be taken to ensure long term operation ~\cite{kininghamCESEL16}.
%While it is still worthwhile to consider how we might build system and power
%supply architectures that support very long lifetime operation,
Before advances in device and sensor technologies are realized,
we expect that the increases in usable energy shown in this work may translate into
sensor nodes that are smaller in size or more capable,
rather than sensor nodes with multi-decade lifetimes.


