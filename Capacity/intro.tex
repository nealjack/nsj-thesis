\section{Motivation and Background}
\label{sec:capacity_intro}

Non-rechargeable (primary-cell) batteries have been
the preferred method of powering sensors for both academic experimentation
and commercial applications. They enable sensors that are easy to design, simple to program,
and reliable to operate until their batteries are exhausted.
However, as we strive towards ubiquitous sensor deployments aimed
at supporting applications such as building automation and industrial monitoring,
the human cost of frequent battery replacement may become untenable.
%especially if it must occur frequently.

With the goals of alleviating battery replacement costs and deploying sensors
in difficult-to-access environments, researchers have explored sensor designs that can
rely solely on harvested energy. Due to declining active and idle power of core
system components, subsisting on the
small amounts of energy available from indoor solar, ambient RF, and thermal
energy sources has become possible in the last
decade. Initially,
%sensors in this space stored the
harvested energy was stored in capacitors because contemporaneous rechargeable
lithium ion batteries required complex charging circuitry
and offered low cycle lifetimes---undermining
the goal of long-lifetime sensors.
%rechargeable Lithium batteries were available, their charging circuitry was
%more complex, higher power
%
%
%Realizing this issue, along with coinciding lowering system power,
%and the
%availability of low-power energy harvesting circuits,
%researchers began
%exploring the possibility of building sensors that rely solely on harvested energy.
%At the time, they considered the various options for buffering this energy, and
%concluded that capacitors were the best option.
%Rechargeable Lithium batteries presented serious lifetime concerns.
%and although they are used now, supercapacitors at the time had significant leakage that could waste
%a significant portion of already limited energy.
%Capacitors, on the other hand, possess a theoretically infinite lifetime, low
%leakage, and high temperature resistance.

%Recent work has also embraced
%supercapacitors, as they offer more capacity at the cost of more leakage, lower
%efficiency, and finite lifetimes.

Unfortunately, capacitors offer
small energy capacity relative to the sensing, computing, communication, and storage tasks
performed by sensors, and systems that employ them must cope with this limitation.
Specifically, capacitor-based systems cannot perform atomic operations that require
more energy than can be stored in the capacitor.
They must intermittently work through non-atomic tasks over
several iterations of starting up, performing some work,
depleting energy reserves, and recharging. Still, significant progress
has been made in making these systems more reliable and programmable.
Progress latching and checkpointing techniques~\cite{lucia2015simpler, ransford2012mementos} enable forward progress through reboots,
special debugging tools~\cite{colin2016energy} can emulate and replay energy state, and finely-tuned
or reconfigurable power supplies~\cite{hesterFlicker17, colinReconfigurable18} increase sensor availability under varying
workloads. Even with these techniques, capacitor-based energy harvesting
sensors are less reliable and more difficult to use than their battery-powered
counterparts.

%Much work in this space has been devoted into
%making these systems more reliable and programmable by introducing progress
%latching, debug techniques, and reconfigurable power supplies, but these
%solutions are complex and have not solved all issues.
Recent trends in technology, including new energy harvesting battery management
ICs and advances in rechargeable battery (secondary-cell) chemistries, suggest that we no longer may be
limited to using capacitors in low power, long-lifetime
designs~\cite{jackson2018reconsidering}. New energy harvesting
ICs are low leakage and offer high efficiency max power point tracking even at low
harvesting voltages and currents~\cite{bq25505, adp5091}. New battery chemistries like lithium
titanate (LTO)~\cite{LTODatasheet} and lithium iron phosphate
(LiFePO\textsubscript{4})~\cite{30mahlifepo} can withstand 4,000-10,000 cycles
before cell degradation~\cite{wangCycle11, omarLithium14} and come in small
(100-700\,mm\textsuperscript{3}), and inexpensive (~\$1 USD) packages~\cite{LTODatasheet, LTODatasheet2}.
%Adding significantly more energy capacity to energy harvesting sensors has the
%potential to mitigate limitations that force intermittency, allowing more
%availability and responsiveness to sensing events as well as enabling more
%intensive workloads.
This work seeks to analyze and characterize the impact of energy
storage capacity on energy harvesting sensor systems.  We define a design
space that is characterized by system capacity, harvesting potential, and
workload,
%UNCOMMENT IF WE ACTUALLY DO THIS:
%and we categorize existing systems within this design space,
identifying where different intermittent computing techniques are necessary and
helpful. Further, we explore the benefits of capacity on energy utilization and
reliability.
%as well
%as to provide insight as to when intermittent techniques are necessary.
Finally, we show the potential gains offered by including a pre-charged, backup
energy store, such as a primary-cell battery, in a sensor's power supply and
examine the impact on lifetime.

%A backup can ensure a minimum, fully
%reliable lifetime.
In our initial analysis, we consider human-occupied indoor environments and
solar energy harvesting sensors.  We find that in these scenarios,
increasing energy storage capacity to an amount much greater than that required
by a sensor's workload eliminates the need for checkpointing and
%that lowering
%sleep power relative to harvester power
significantly reduces the impact
of finely-tuned and reconfigurable power supplies. Eliminating the necessity
of these techniques further increases operating efficiency because the techniques
themselves draw considerable power.

To analyze the impact of capacity under variable harvesting
conditions or variable workload, both of which are common in real-world
environments, we develop a numerical model that
%explore the effect of energy capacity and
%backup storage on sensor performance,
utilizes energy traces and several workloads that are representative
of real hardware.  We use this model in the context of solar energy harvesting,
and find that sensors with higher capacity, on the order of that of a battery,
capture and utilize 1.3-2.6x the energy of systems with capacitor-sized energy
stores. This energy allows systems to perform near 100\% of scheduled tasks on
time compared to only 20-80\% for systems with insufficient energy capacity to
continue operating through energy droughts, or periods in which there is not
sufficient harvestable energy to sustain operation.

We find that the inclusion of a backup energy store like a primary-cell
in energy harvesting systems can eliminate nearly all drawbacks
of energy harvesting designs.
Backup energy can be used to retain state across reboots, cold-start harvester
front ends to increase their efficiency, and boost the availability of a sensor
when harvested energy is depleted. While a backup energy
store does come with a finite lifetime, our model predicts that
hybrid designs performing all of these operations
achieve 2-4x the lifetime of primary-cell only designs. A backup energy source
that is only used for state retention and harvester cold-start would offer
longer lifetimes.

%Even when using a primary-cell for all of
%these operations
%
%the model to show that an additional primary
%to perform 100\% of expected tasks on time, compared to only \hl{40-80\%}
%for systems that utilize capacitor-sized energy storage and cannot store
%enough energy to subsist through periods of energy drought.
%reliability significantly increases with
%capacity, relaxes the need for intermittent techniques, and enables
%significantly more intensive computation and operations.
%A modeled sensor with increased capacity is able perform 100\% of expected
%tasks on time, compared to \hl{40-80\%} for a system that utilizes energy
%storage on the scale of a capacitor. We also find that energy harvesting in
%tandem with a backup source achieves 4-6x the lifetime of a
%primary-cell only design.

Based on our design exploration, we implement a solar energy harvesting platform,
\name, to enable autonomous lighting control for indoor spaces. It has
a 20\,mAh secondary-cell for more energy capture
%during periods of high harvesting potential
and a backup energy store to ensure
state retention and high reliability even during long energy harvesting droughts.
%(such as
%when the building shuts down for multiple weeks during an academic break).
%We use the results of this model to implement optimal energy capacity in an
%energy harvesting platform named \name, which is designed to enable
%high-fidelity, autonomous lighting
%control for indoor spaces, with the goals of long lifetime and high
%reliability.
We present the new and low-power components selected for its
design, demonstrating the viability of the design point and
power measurements of
these components,
which are used as the basis for the standard workloads that we
define and model.  To validate the model,
we deploy \name along with battery-powered and intermittent systems and
compare their performance to that predicted by the model. We
show that lifetime estimates, the frequency and timing of charge-discharge
cycles, and expected number of transmissions are similar to that of the
deployed systems.

When we began the design of \name, we initially intended to use capacitors as
the rechargeable energy store.  At the time,
%we believed the
the assertion that batteries were expensive, bulky, and had
extremely limited lifetimes was prevalent
~\cite{hesterNew17, hesterTragedy15, hesterFlicker17, hesterTimely17, colinReconfigurable18, luciaIntermittent17}.
Upon further examination, we find that
these claims are no longer true for many applications due to technology
improvements, and the gains provided by increased energy storage
capacity are numerous.  Greater rechargeable capacity allows a system to obviate
the need for intermittent techniques and harvest more energy. More collected
energy translates to higher reliability and capability.  The addition of a
primary-cell allows an otherwise unreliable system to operate without
interruption for years or decades, or to support state retention and eliminate
expensive cold start.  With these advances, we envision future energy
harvesting sensors which are not limited by the visions of immortality but are
instead long-lived, reliable, and capable of enabling interesting and useful
applications.


%We do not argue the current
%possibility of including more storage on sensors,
%although recent work \hl{cite
%enssys} has suggested new battery technology and techniques can offer
%significantly more capacity than current capacitors and supercapacitors.
%
%By using batteries in energy harvesting sensors, we believe that we can
%create extremely long-lived, available, and responsive sensors.
%These sensors will not need to micro-manage
%their energy state and can instead adapt their lifetime and energy
%usage over the course of weeks, months, or not at all. This push
%toward available, responsive, and capable sensors will enable more standard
%programming models, energy management techniques, and ultimately useful
%applications to be built on top of dense and ubiquitous sensor deployments.





%Some key thoughts on framing
%    - We need to make sure that it comes through this is targeting what
%    we define as the ``common'' use case.
%        - We should be clear about what we consider the common use case,
%        and I think it's obvious from our current direction that this should
%        be indoor/building sensing (Although we could consider trying
%        and outdoor scenario?? higher harvesting potential better
%        drives the point of having a rechargeable battery.)
%    - When we talk about reliability it should be clear how we are defining
%    it and why it is important. Really it is the important metric (along with
%    lifetime). I'm going to try to look for some citations about
%    how humans respond to failure. My sense has always been that even low
%    failure rates are really frustrating, but we should have some data to back
%    this up.
%    - Reliability also means liveness. This is the most frustrating thing
%    about deploying intermittent sensors! is it dead or just not getting
%    energy? It's hard to pre-emptively catch failure if you can't test
%    failure until it's too late.
%    - There is this idea that we designing a more usable node by using
%    batteries. Rather than micro-optimizing every transaction batteries
%    allow us to macro-optimize energy and applications over time. This could
%    mean adapting your application without missing your deadline (i.e occupancy
%    sensors that only sense lack of occupancy every 10 minutes instead of every
%    minute).  It's unclear to me how much this should come through. I definitely
%    think that intermittency pushes functionality away from the edge, and
%    reliablity pushes it toward the edge, but that is its own argument so we
%    probably have to be careful.
%    - How much of a role does permamote play into this story. Currently it is relegated
%    to the platform that came out of these design decisions. This is not a platform
%    paper. It could arguably be a platform paper. What is the next version of
%    the sensor node everyone uses? Maybe it is the sensor that integrates
%    this work on energy harvesting. The current paper is not framed this way though
%    so we should decide sooner rather than later if that's a good idea. I generally
%    think that platform papers are more hit or miss.
%        - If permamote plays a role we should absolutely include its components
%        selection matrices. Not the trendlines, but the lines motivating that
%        these really are the best parts and by choosing them we get significant (2-4x)
%        lifetime improvements over designs that are even a few years old. (AKA stop using the RF233).
%When someone is done reading this paper what do we want them to take away?
%    - Using batteries primary and secondary-cells
%    on sensor nodes prioritizes the common case
%    (long lifetime, normal environment, high reliability) which we are currently leaving behind.
%    - For very little physical volume we can completely avoid intermittency for a very
%    long time and that is a worthy tradeoff. Even if your primary-cell dies
%    you are no worse than an intermittent node.
%    - Batteries are not as bad as everyone thinks and they are getting much
%    better in time.
%    - Something about node size? This was toyed at in the grafting paper, and
%    I think we have the chance to really nail it. At what point can do the following tradeoffs occur:

