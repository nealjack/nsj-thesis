\chapter{Introduction}
\label{chap:intro}

The past two decades have witnessed the proliferation of low power, autonomous, and wirelessly connected devices.
The development and deployment of these devices were inspired by the vision of ubiquitous computing: computers that are pervasive in our environments, providing information about and intuitive control over our environments in such a way that they disappear into the background~\cite{weiser1991computer}.
While initially inspired by this vision, today's internet of things (IoT) has failed to fully achieve it.
There are many problems standing in the way of this vision, but one of the most pressing and fundamental problems is power.
Low power embedded systems have experienced exponential increases in computational and power efficiency in the last two decades. Across the board, components used to build wireless systems have increased in power efficiency by one to two orders of magnitude.
Our ability to process and infer meaning from vast amounts of collected data has also improved by orders of magnitude, with approachable and powerful frameworks for developing effective machine learning models for classification and detection.
However, the methods of collecting and allocating energy to power wireless sensing systems have not improved at the same rate.
So, while today's small wireless sensing devices have become smaller, more efficient, and more capable of powerful and complex sensing and data processing, their capabilities are limited by the power and energy available to them. 
%The design, implementation, and deployment of low power wireless sensors has been studied and improved significantly over the past two decades. Despite this, wireless sensor networks can hardly be described as ubiquitous and pervasive in our environment. Sensor deployments are only commonly used with specific applications; they are from resembling the pervasive and invisible computer imagined by Mark Weiser and Xerox PARC~\cite{weiser1991computer}.

\section{The Power and Energy Dilemma}
Today, if you wanted to build a wireless sensor, perhaps the most important step in the design process is determining how to allocate or capture the energy to power that sensor. 
Nowadays, there are many options for energy storage--- capacitors, supercapacitors, batteries, and even more options for energy harvesting, including photovoltaic panels, thermoelectric and piezoelectric generators, and microbial fuel cells~\cite{yervaGrafting12,campbellThermes14,campbell2018energy,josephson2020farming}.
This results in a vast system-level design space that can be difficult to navigate.
To reduce the design space problem to something comprehensible, it is reasonable to consider existing industrial and research designs as templates for a design.
We may look to modern commercial sensors for inspiration, like the Google Nest temperature sensor~\cite{googleNestTemperature}, or to a modern research device like the Flicker platform~\cite{hesterFlicker17}.
While these two examples use similar sensors, microcontrollers, and radios, and are designed for a similar sensing purpose, they exhibit disparate differences in their power supply design.
These two different designs represent a fork in the road for wireless system design.

Like most modern commercial sensors, the Nest temperature sensor utilizes a non-rechargeable battery for an energy source. 
Batteries have been the preferred method of powering sensors for the last twenty years. And for good reason: they are simple to use and provide reliable and predictable power.
However, a battery provides a finite lifetime and battery or sensor replacement is inevitable.
Wireless sensor lifetime is a first order concern, as the cost of frequent maintenance and battery replacement is clearly untenable as the number of wireless devices grows to trillions of devices. 
The act of maintaining what should be an invisible sensor renders it visible due to the annoyance of changing a battery.
Changing a single battery is not usually an issue. 
Still, it is not uncommon for people to simply ignore a dying device like a smoke detector, opting to remove the battery instead of replace it when low. 
Changing the battery of hundreds or thousands of devices within a building is a significant and costly undertaking.

Recognizing the problem with short, finite lifetimes, researchers have forged ahead with designs that aspire for immortality.
For a decade, many researchers have abandoned batteries completely and instead built systems that harvest energy to power themselves. 
Sensors built with this technique possess indeterminate lifetimes. The cost of immortality is steep, however.
These systems generally require significantly more software and hardware design complexity, and operate unpredictably and without any quality of service guarantees.
Due to this, energy harvesting systems without batteries have seem limited adoption by industry, despite significant excitement and a large corpus of work by researchers. 

%Currently, as a community of wireless system designers, we lack a shared understanding regarding how to actually approach system-level power supply design. 

%Despite these flaws, researchers have touted batteryless systems as the design pattern for future sensing applications~\cite{hester2017future}. 
%To be truly pervasive, systems must only require minimal maintenance. While some applications are important enough to justify some routine maintenance, many are only realistic and cost-effective if sensors can be deployed and forgotten for long periods of time.
%The continual improvement of power efficiency across all wireless sensor components has resulted in battery-based sensors that can persist for decades, as well as batteryless sensors that can function with minute amounts of ambient energy.
%To achieve long lifetimes, battery-based sensors must sacrifice size to employ large batteries, while a purely energy-harvesting system must sacrifice any guarantees of availability and quality of service as their energy income may not be guaranteed.

\section{The Difficulty of Gathering Data}
In reality, neither option on its own, battery or energy harvesting, is an entirely satisfactory solution for many sensing applications. 
Modern data gathering applications often simultaneously demand longevity and consistent operation.
It is often not worth the cost of deployment and frequent maintenance if sensors only last a short while, or if the data they produce is not ultimately useful for the end goal of the data gathering application.

Traditional environmental sensing applications like wildlife monitoring and tracking requires consistent environmental and location data gathering over seasons and years~\cite{mainwaring2002wireless,juang2002energy}.
Asset tracking applications have to be able to reliably locate and track commodities throughout an entire supply chain, including weeks or months in a warehouse, or on store shelves~\cite{williotpixel}.
Infrastructure monitoring applications must consistently and reliably detect vibrational anomalies or corrosion throughout the multi-decade lifespan of a bridge or a building~\cite{afanasov2020battery,jagtap2021repurposing}.
All of these example applications require consistent and reliable data gathering over long periods of time.

New applications are made feasible with new and improved technology, but they would be still be limited by either battery or energy harvesting options.
As an example, person detection and room occupancy counting is a useful metric for building lighting and HVAC control.
Having reliable occupancy can increase efficiency as the building management system can provide light and comfortable heating tailored to where people exit within the building. 
%gathering visual image data to train a model to perform object detection is a difficult and complex task for low power sensors.
%Traditionally, this task has been a human one, and the ability to gather sufficiently large datasets has been relegated to large companies like Google and Facebook. These companies rely on their large userbases to produce images at a scale sufficient to build datasets to train their models.
Capturing, manipulating, and transmitting image data requires significant computational, memory, and networking resources. 
Even with advances in embedded computing and CMOS imaging, these resources are generally costly in power compared to the traditional, simpler one-dimensional data collection of traditional environmental sensors~\cite{mainwaring2002wireless}. 
If designed with a battery, such an imaging system would possess a short lifetime. If designed to be energy harvesting without a battery, the sensor can provide no guarantees for when or how many images are captured, potentially failing to perform the designed purpose: reliably capturing when people are present and how many are occupying a space. 

This choice between a battery or batteryless energy harvesting solution is a deceiving one.
This choice is predicated on the assumption that a sensor design must be one or the other, battery-based or energy harvesting.
Instead of thinking of wireless sensor power supply design as adhering to a predetermined class or pattern of design, designers should consider an application's requirements and how those requirements impact the design space.
Many recent energy harvesting systems are built with the assumption that batteries can not be trusted, both non-rechargeable and rechargeable
~\cite{hesterNew17, hesterTragedy15, hesterFlicker17, hesterTimely17, hester2017future, colinReconfigurable18, luciaIntermittent17, yervaGrafting12, majid2020continuous}.
This belief has led designers to make arbitrary design decisions regarding the type and sizing of components within their power supply in order to achieve minimally feasible batteryless designs.

%The operation of a batteryless system is inexplicably tied to the availability of harvestable energy.
%This is due to the limited charge capacity provided by non-battery options. Batteryless systems generally utilize capacitors or supercapacitors for energy buffering instead of batteries, and are orders of magnitude less energy dense.
%This results in an energy uncertainty; these devices often run out of energy and lose volatile state at potentially inopportune and unpredictable times. 
%With such intermittently powered systems, energy intensive operations require either careful manual tuning to operate within their limited energy envelope~\cite{yervaGrafting12,campbellEnergy14,debruin2013monjolo}, or complex software techniques to enable eventual forward progress across reboots~\cite{ransford2012mementos,maeng2017alpaca,hesterTimely17}.
%Batteryless system software that ensures forward progress must manage energy state on short timescales and save and restore non-volatile state in ways that avoid consistency problems. 
%These problems are challenging to develop for and reliably debug and researchers have responded by introducing software frameworks and compilers that provide safety guarantees and hide these complexities from programmers. 
%While seemingly attractive, these abstractions hide the realities of batteryless performance from programmers, instead providing a notion of non-stop, continuous, and eventual progress. 
%In reality, these systems are only as continuous as their source of harvested energy, and programmers and users are left without a method of reasoning about system performance.
%
%%To enable forward progress, researchers developed software solutions to help generalize computing on these platforms through software techniques like checkpointing~\cite{ransford2012mementos}.
%  %As the programs got more complex, and the hardware more sophisticated, it became clear that adding bits of capacitance could help, but only for so long, since the energy density was so low
%
%As the energy intensity of sensing and computational workloads increase, it becomes necessary to increase the storage capacitance of these systems by adding larger or additional capacitors, or utilizing newer supercapacitor technology. 
%This increased capacitance results in longer charging hysteresis, in which it takes a long time to charge the storage capacitor to a sufficient voltage level, limiting the system's responsiveness.
%To address this, researchers have explored complex hardware solutions to manage hysteresis, and additional complex software to manage this hardware~\cite{colinReconfigurable18,hesterFlicker17}.
%
%After all this complexity, batteryless systems are still unable to perform sensing tasks with high availability in the common case where harvestable energy is not consistent.
%Batteryless systems possess a single \textit{fundemental} limitation: their energy capacity.
%This limitation discounts their applicability in use cases that demand any estimates or guarantees of availabilty and quality of service.
%
%Researchers developing energy-harvesting systems have continued to argue that batteries (both non-rechargeable and rechargeable) can not be trusted
%~\cite{hesterNew17, hesterTragedy15, hesterFlicker17, hesterTimely17, hester2017future, colinReconfigurable18, luciaIntermittent17, yervaGrafting12, majid2020continuous}.
%%The prevailing belief that batteries offer very limited lifetimes and are difficult to manage and charge has led to their abandonment by batteryless system researchers. 
%The assumed superiority of batteryless design has resulted in a constrained and stagnated design space, in which the terms energy-harvesting efficacy is a severely hampered, and the inherent unreliability and requisite complexity of the design regime dissuades serious and wide-spread adoption by industry.

%The design, implementation, and deployment of low power wireless sensors has been studied and improved significantly over the past two decades. Despite this, wireless sensor networks can hardly be described as ubiquitous and pervasive in our environment. Sensor deployments are only commonly used with specific applications; they are from resembling the pervasive and invisible computer imagined by Mark Weiser and Xerox PARC~\cite{weiser1991computer}.
%
%While there are many problems that stand of the way of this vision, this dissertation's focus is on how to power the computer of the 21st century. 
%%To be truly pervasive, and invisible, a system must reliably provide sufficient utility to justify the cost of its widespread implementation and deployment, and require such little maintenance 
%%to the point that its existence can be ignored.
%To be truly pervasive, systems must only require minimal maintenance. While some applications are important enough to justify some routine maintenance, many are only realistic and cost-effective if sensors can be deployed and forgotten for long periods of time.
%The current direction of wireless sensor design research is
%primarily focused on maximizing the longevity and minimizing the maintenance of these systems, but in this pursuit have entirely sacrificed utility and reliability. Researchers have abandoned the reliability and predictability offered by batteries, and touted batteryless systems as the design pattern for future sensing applications~\cite{hester2017future}. 
%
%Long-lifetime and low maintenance sensors \textit{are} a requirement for pervasive sensing, as the maintenance and battery replacement of the eventual trillions of wireless devices is clearly untenable~\hl{cite}. 
%However, 
%the vision of ubiquitous sensing can not succeed without useful and reliable devices. This dissertation explores the design space of wireless sensor power supplies in order to achieve designs that offer long-lived and reliable operation. 
%%How have so many system builders arrived at the conclusion that this design point is the future of sensing? 
%
%\section{A Brief History of Energy-harvesting Sensing}
%In order to understand the current state of energy-harvesting wireless sensor design, it is important to revisit the history of their development. The inception of wireless sensor networks became possible around the turn of the millennium, when improvements in microcontroller and short range radio technology resulted in efficient components requiring very low active and idle power. This enabled heavily duty-cycled systems that could reliably persist over the course of months to a year on battery power~\cite{mainwaring2002wireless,tolle2005macroscope}. 
%For longer deployments, or in applications where the size and weight of a non-rechargeable battery was untenable, researchers also developed energy harvesting solutions, the majority of which utilized outdoor solar power and rechargeable batteries~\cite{juang2002energy,raghunathan2005design}.
%
%Around the same time, researchers at University of Washington and Intel Research Seattle began experimenting with computational RFIDs (CRFIDs) to create battery-free sensors built around the WISP platform~\cite{sample2008design}. 
%These researchers noted the limited storage of non-rechargeable batteries, and the limited cycle lifetime of rechargeable batteries. 
%Instead of batteries, these systems utilize small capacitor-based energy buffers that are able to store just enough energy to complete a small atomic task, be it operating a sensor, transmitting a packet via RFID backscatter, or performing some amount of computation. 
%While the longevity of CRFIDs are not limited to to the lifetime of a battery, they require the proximity of an RFID reader to provide power and bidirectional communication. 
%Since the development of CRFIDs, researchers have extended the technique beyond RFID to systems with active radios and other harvesting methods, including photovoltaic, piezoelectric, and thermoelectric harvesting~\cite{yervaGrafting12, debruin2013monjolo, hesterFlicker17, colinReconfigurable18, nardello2019camaroptera}. This approach has since become very popular, with many researchers convinced that it is the future for wireless sensor design~\cite{hester2017future}. This has resulted in the confluence of the terms "energy-harvesting" and "batteryless" in the community. 
%
%
%\section{A Research-Industry Rift}
%There is an implicit disagreement between research and industry when it comes to powering wireless sensors. 
%While many researchers are convinced that batteryless design represents the future of wireless sensors, this confidence has not extended to system builders in industry, despite many years of research and development. 
%To achieve reliable and predictable operation, 
%industrial
%~\cite{emersonRosemount,GEInsightMesh,honeywellOneWireless}.
%and commercial sensors 
%~\cite{ecobeeSensor, honeywellThermostat, lutronSolutions, googleNestTemperature, hueSensor} 
%have, for the most part, rejected energy harvesting and continue to utilize the same design pattern as the first wireless sensors built 20 years ago: a non-rechargeable battery~\cite{polastre2005telos, hill2002wireless}. 
%Energy-harvesting has been utilized in industry, but is generally limited to obvious applications with plenty of harvestable energy, like outdoor photovoltaics. 
%
%This reality begs the question: why is it that industry has largely ignored energy-harvesting and batteryless design? 
%%I argue that the primary reason is the fact that batteryless systems
%%For the last decade, many platform builders in the research community have eschewed batteries and embraced energy-harvesting techniques for powering sensors with the lofty goal of building immortal, maintenance free sensors~\cite{hester2017future}.
%Despite more than a decade of batteryless energy-harvesting system design, implementation, and improvement, this body of work has failed to materialize serious solutions for industrial and commercial applications, nor has it ushered in a golden age of long-lived, distributed, and ubiquitous sensing and computing.
%The reality is that these designs mandate a level of complexity, unpredictability, and unreliability that hinders their adoption in a marketplace that prizes simplicity, consistency, and availability.
%
%The design disparity between industry and research 
%points to a lack of understanding about how to approach designing energy-harvesting systems that can achieve the requirements of real-world applications. This dissertation seeks to provide clarity and serve as a design guide for building low power wireless sensors that utilize energy harvesting to elongate their lifetimes while maintaining high reliability and predictability.


\section{Thesis Statement}
An analytical and simulation driven design framework provides better guidance for system-level power supply design than current energy harvesting intuition. We can use design points determined through design space analysis and simulation to build systems that provide higher availability, lower latency, and long lifetimes, all without the need for complex techniques for managing intermittency.

\section{Contributions of this Dissertation}
This dissertation attempts to address system-level power supply design in the context of real application requirements.
Various parts of this dissertation have been published in part at ENSsys'18~\cite{jackson2018reconsidering} and IPSN'19~\cite{jackson2019capacity}. Content from these publications are included throughout the entire document, including \cref{chap:intuition}, \cref{chap:capacity}, and \cref{chap:battery}. 

We begin with an examination of existing power supply designs and methods for providing energy to sensors. In \cref{chap:background}, we discuss traditional battery-based sensor design as well as the history of energy harvesting in wireless sensor research.
We explore the design and trade-offs of batteryless system design, a newer class of energy harvesting sensors that forgo the use of batteries as energy storage in favor of capacitors and supercapacitors.
We identify the strengths and weaknesses of both battery-based and batteryless design points within the context of real applications based on previous work.
From this exploration, we identify several common design considerations.

In \cref{chap:intuition}, we utilize these considerations to develop system-level design constraints and heuristics.
We consider and compare the efficacy of batteries and energy harvesting and identify inflection conditions where either option excels over the other. We identify the benefits of a rarely used hybrid architecture that employs both options and consider the performance and lifetime implications of a system that utilizes it.
Next, we define design constraints regarding energy income and energy buffer size that partition the design space into several regions to reason about the necessity of batteryless an intermittent techniques. 
%Depending on income and buffer size, a design may be feasible or infeasible, and require or not require complex software and hardware techniques to ensure operation in adverse, low power environments.
We finish this chapter with a deep dive into the effect of rechargeable energy capacity on the performance of an energy harvesting system. From this exploration, we utilize a simple model of energy capacity to develop novel and easy to use design heuristics. These heuristics determine the minimum sufficient sizing for energy capacity for energy harvesting systems. 

We expand upon the simple model and heuristics developed in \cref{chap:intuition} and build an wireless sensor energy state simulation in \cref{chap:capacity}.
We use this simulation to verify our capacity sizing heuristics, consider additional design variables, and examine the performance of the hybrid battery and energy harvesting architecture.
From the results of our energy simulation, we identify the benefits in energy capture and end system performance when rechargeable energy capacity is increased. From these results, we conclude that energy harvesting system performance can be increased by 1.4--2$\times$ if capacity is increased several orders of magnitude more than is commonly offered by batteryless system's capacitor or supercapacitor energy buffers. 

In \cref{chap:battery}, we explore the design space of energy capacity, identifying the classes and options for charge and energy buffers. We compare capacitors, supercapacitors, and batteries, and examine common arguments made against the use of rechargeable batteries by batteryless researchers.
We find that many of these arguments are unsubstantiated when examined quantitatively and conclude that small rechargeable batteries are the superior method for providing an energy dense buffer for low power applications.

We consider the heuristics, tools, and conclusions from \cref{chap:intuition,chap:capacity} to implement sensors for two wireless sensing applications in \cref{chap:impl}. 
These applications include illuminance sensing for automated lighting control applications as well as image-based person detection and counting for accurate space occupancy measurements, both of which require consistent and long-lived sensing. 
Both sensor designs utilize the hybrid architecture identified in \cref{chap:intuition}, and the utilize heuristic and simulation tools developed in \cref{chap:intuition,chap:capacity} to size their power supply components to achieve high energy capture and system performance. 
We implement and utilize our designs to evaluate and verify our energy simulation developed in \cref{chap:capacity}. 
We evaluate an existing batteryless image sensing platform, quantify its performance using simulation, and identify design changes that could be made to improve performance.
From these design changes, we build a new indoor image sensing platform that can capture image data with high consistency and availability and still offer a multi-year lifetime. 

Finally, in \cref{chap:conc}, we summarize the contributions of this dissertation and identify several new and improved primitives that would enable more efficient and autonomous energy harvesting wireless sensors. 

