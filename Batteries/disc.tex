\section{Discussion}
\label{sec:disc}
%The improvements that modern batteries and management methods provide over
%older battery technology suggests a revisit of the use of
%batteries in energy harvesting sensors.  In many common applications and
%environments, we believe batteries will provide substantially more energy
%capacity and thus availability and responsiveness, alleviating many of the
%problems that stem from limited capacity inherent to intermittent systems.

For applications that involve extreme temperatures,
the high possibility of mechanical damage,
and some scenarios that entirely prevent any sensor
replacement, we expect that capacitor-based intermittent systems remain the best and perhaps
only option.  These include applications such as simple and
inexpensive satellites that must exist in the cold of space,
sensors embedded in concrete that are expected to monitor buildings over decades,
or sensors meant to monitor inside of the human body. While there are some methods
to make batteries function in these environments and ongoing
efforts to continue to improve the properties of batteries such that they
might be sufficient for these applications, these are the places in which
a capacitor-based energy store has a great advantage.

We would like to reemphasize however that many applications do not take place
in these conditions. The instrumentation of our homes, offices,
factories and most of our outdoor spaces will not stress the properties of
batteries and cause them to fail. These spaces will be occupied by people
who can replace sensors at a rate similar to the rate at which we replace
the rest of our furniture, appliances, and technology. Most importantly, these
human-facing applications will see large benefits from the increase in availability
and responsiveness that batteries have the potential to provide.

%we don't expect these efforts
%to be mature for many years.
%For such application, capacitor-based systems may prove to
%be the best option.
%Batteries and supercapacitors are temperature
%sensitive when compared to ceramic and tantalum capacitors.
%Upon damage or abnormality, even newer, safer battery chemestries may heat up uncontrollably.
%Notwithstanding these possible inadequacies, batteries should not be entirely
%discounted for such applications. There may be additional methods in which to enable their use.


%we expect most sensors to occupy
%environments inhabited by people, which are comfortable for battery based sensors, and we believe encompass the majority of sensing applications.

%For
%example, in low temperatures, the greater capacity afforded by batteries could
%enable the occasional use of heating elements to improve efficiency and
%lifetime.  Additionally, defensive case design could be prioritized to protect
%the battery from mechanical abuse.
%The decision to use capacitors for extreme
%applications is not a clear cut one, as


