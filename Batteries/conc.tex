\section{Conclusions}
\label{sec:conc}

After a decade of designing intermittent systems and tackling their issues, it
is time to reevaluate the necessity of capacitor-based storage in the face of technology improvements.
New, modern battery technologies and management techniques allow for high
capacity, long lifetime sensor power supplies without the costs commonly
associated with older battery chemistries. We believe their integration into
sensor designs will lead to more capable devices with greater availability.
Battery-based sensors will ultimately enable dense deployments and autonomous applications in
our occupied environment.
%that have the potential to enable dense and pervasive sensing applications
%in industrial automation, building
%control
%the kinds of automation and applications promised
%since sensor networks were conceived.

%When we started the design of \name, we initially intended to use
%capacitors as the rechargeable energy store. This idea was predicated on the
%assumption that capacitors were, by many metrics,
%a better energy storage option than
%batteries.  At the time, we believed many of the qualitative claims made by the
%sensor networking community: that batteries were inefficient, fragile, and had
%extremely limited lifetimes. Upon examination, however, we find many of these
%claims are unsubstantiated when designing sensors for the common case,
%and that the gains provided by batteries are numerous. Secondary cells can
%provide larger energy buffers, allowing a system to collect more
%energy. More energy collected translates to higher availability and capability.
%The addition of a primary cell allows an otherwise intermittent system to
%operate without interruption for years or decades. In retrospect,
%we see a large body of work that is
%devoted to making intermittency less painful by proposing complex
%software and hardware solutions, when instead intermittency could have been
%avoided altogether for most applications by just using batteries.

%Instead, many of the headaches of
%intermittency can be completely avoided by just using batteries. Going forward
%as a community, we need to make sure to quantitatively evaluate our assumptions
%and clearly define and evaluate intended use cases.
