% ABSTRACT

%Battery-powered sensor nodes have long been the standard for simple and
%reliable sensor deployments, however they can suffer from short lifetimes and
%high maintenance costs associated with battery replacement.
%

For the past decade, the status-quo for energy harvesting sensors
has been to buffer small amounts of energy in capacitors to
intermittently work through a sensing task. While using capacitors for storage
offers these systems indefinite lifetime, it comes at a cost --
%While these systems avoid
%batteries, which are often dismissed as expensive, failure-prone, fragile, and
%temperature sensitive,
they must tolerate the decreased availability, lower energy utilization, and more complex programming models
inherent to a volatile, intermittent design.
We argue that many of these problems stem from insufficient energy storage and
could be eliminated with the use of batteries.  Recent advances in rechargeable battery
technology weaken the
historical arguments against their use. We believe that using batteries
in energy harvesting sensors will push us closer to a class of reliable,
general purpose devices that can better serve human-centric sensing applications
than their capacitor-based counterparts at the cost of having a finite, but long, lifetime.

%
%and new sensor designs that employ them
%will benefit greatly from the increased energy storage afforded by rechargeable
%batteries.
%
%While batteries do have a finite, but long, lifetime, their
%capacity allows future sensors to avoid
%costs and single-purpose designs associated with perpetual but
%intermittent sensors.
%poor availability, painful
%programming,

%This work bridges the gap between these two design points, advocating
%for the use of energy harvesting to extend the lifetime of
%a sensor node, exploiting rechargeable batteries to increase the
%utilization of harvestable energy, and relying on non-rechargeable batteries
%to ensure availability and avoid intermittency throughout the sensor's lifetime.
%We support this use of batteries by showing that many of the criticisms lodged against
%them are unfounded, outdated, or easily mitigated for the great majority of
%use cases. By modeling the charge-discharge patterns of
%energy stores in indoor harvesting
%environments, under both periodic and event-driven workloads, we show that this design
%can capture 1.5-2.2x the available energy of capacitor-only
%designs and achieve 4-6x the lifetime of primary-cell only designs, all
%while avoiding the poor availability, painful programming models, and special-purpose
%hardware platforms associated
%with intermittency.

%We support these claims by modeling the charge-discharge patterns of energy
%stores placed in real energy harvesting scenarios under periodic and event
%based workloads, and we defend our use of batteries by showing that
%the claims against them are either unfounded, outdated, or easily mitigated
%for the vast majority of real use-cases.
%
%
%
%By modeling the charge-discharge patterns of energy stores placed in real
%energy-harvesting scenarios under periodic and event-based workloads, we show
%the performance increases associated
%
%of
%the sensor node. To support these decisions we model the charge-discharge patterns
%of energy stores when placed in real, energy-harvesting scenarios under periodic
%and event-based workloads,
%
%
%
%By forgoing the high capacity of
%batteries however, these everyla
%
%
%however they inherently trade-off
%availability
%
%Harvesting ambient energy to power distributed sensor nodes has the
%potential to extend their lifetimes and increase their performance compared
%to non-harvesting, battery-powered designs. Sensor platforms presented
%in prior work take advantage of this harvested energy by temporarily storing it in
%capacitors before using it to perform a small number of operations, intermittently
%working through a sensing task. Some authors
%
%Energy harvesting has the potential to extend the lifetime and increase
%the available
%
%however current
%energy harvesting architectures
%
%Energy harvesting has the potential to power distributed sensing nodes,
%extending their lifetimes, increasing the energy available to
%perform sensing tasks, and reducing the maintenance cost of a sensor
%deployment. In the past five years a trend has even
%emerged pushing for ``perpetual'' batteryless sensors,
%supported by claims that batteries are fragile, toxic, temperature
%sensitive, and prone to long-term cycling failure. Alternative
%designs temporarily store harvested energy in capacitors before performing
%a small number of operations, intermittently working through a sensing task.
%These intermittent designs, however, are inherently prone to low availability
%for many workloads, present users with non-standard programming models to mask
%their intermittency, and make poor utilization of the harvestable ambient
%energy.
%
%We argue that utilizing batteries, both rechargeable and non-rechargeable cells,
%can solve these problems for the vast majority
%of real workloads and deployment scenarios while still providing
%the key benefits of energy harvesting.
%By analyzing the availability, lifetime and harvestable energy utilization
%for a variety of energy storage systems we motivate the need to use
%batteries, and we welcome their use by
%showing that many claims against them are unfounded, outdated or easily mitigated.
%To perform the analysis, we model the charge-discharge patterns of
%energy stores when placed in real, energy-harvesting
%scenarios under periodic and event-based workloads, and we evaluate this
%model by comparing it to implementations of several points in the design space.
%We show that for reasonable energy storage sizes, workloads, and
%energy availability, sensor
%nodes utilizing batteries along with energy harvesting can capture \hl{6x} the available
%energy of capacitor-only designs and achieve at least \hl{4x} the lifetime of
%primary-cell only designs, all while avoiding the poor availability and painful
%programming models associated with intermittency.
