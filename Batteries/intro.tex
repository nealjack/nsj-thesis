\section{Introduction}
\label{sec:intro}

%Since the early 2000s, sensor network power supply design has
%experienced two significant design philosophies: using non-rechargeable energy
%stores, and relying partly or fully on energy harvesting from the ambient
%environment.
%
%The first sensor networks used nodes that rely entirely on large
%non-rechargeable batteries for energy~\cite{mainwaring2002wireless,
%polastre2005telos}. These large cells provide reliable power but ultimately a
%limited lifetime. When considering dense
%deployments of these devices to enable applications like industrial monitoring
%and building automation, the cost of battery replacement with such frequency is untenable.

The first sensor network nodes relied on non-rechargeable batteries
(primary cells) for energy storage~\cite{mainwaring2002wireless, polastre2005telos}. These primary
cells provided reliable power that enabled quick
initial
tests and deployments. However the drawbacks of this design quickly became
clear---the short sensor node
lifetimes achievable with reasonably sized primary cells required frequent
battery replacement for nodes to remain operational, and as a result, researchers
started looking to energy harvesting as an alternative power source. This began
with outdoor solar harvesting, but soon transformed as researchers
started to leverage lower power components and new
programming models to enable energy harvesting
sensor nodes with significantly less capable power supplies.

Starting with directed, RFID-based sources~\cite{smithWirelessly06}, and moving
to less reliable energy sources like indoor solar and vibration~\cite{gorlatovaPrototyping10, yervaGrafting12, campbellEnergy14},
these designs fueled a vision of perpetually
operational sensor nodes powered by the ambient environment.
At the time, capacitors were the only easy, off-the-shelf method
of storing energy from these ultra-low current power sources.
Supercapacitors had high leakage that could waste a significant portion
of the already limited energy.
%Supercapacitors can have high leakage immediately after charging that
%wastes a significant portion of the already limited energy.
Lithium batteries were difficult to manage, as low power battery
management and charging ICs were not readily available.
%and there were not low power battery management
%ICs, often required for the practical use of lithium batteries.
Additionally
the theoretically infinite lifetime of
capacitors further supported the goals of perpetual sensing, especially
compared to the life-cycle limits of batteries. Over time, other weaknesses of
batteries compounded the arguments, including their temperature sensitivity,
physically large form factor, and potential danger for both the user and
the environment.

%mostly used capacitors to store the small amounts of energy
%they could harvest

%Many researchers, realizing new technology, have
%explored the potential to use energy harvesting methods to power sensors in order to
%lengthen lifetimes and avoid the replacement of non-rechargeable batteries.  As
%recently as a decade ago,
%researchers began to explore using energy harvesting in indoor environments
%using small, unobtrusive devices that harvest energy from ambient
%lighting, the thermal gradient between surfaces, or vibration from
%human motion~\cite{yervaGrafting12, campbellEnergy14, campbellThermes14}.

%To buffer the harvested energy, these early systems opted
%to use capacitors, both low capacity tantalum and ceramic, as well as higher
%capacity supercapacitors (supercapacitors), rather than rechargeable batteries~\cite{yervaGrafting12, colinReconfigurable18, hesterFlicker17}. The decision to
%eschew batteries, even though they offered substantially more energy capacity,
%was predicated by the many undesirable qualities that defined batteries at the
%time, most notably that they were expensive, possessed very short cycle lifetimes (the number
%of cycles until a battery's capacity is significantly
%degraded), were temperture-sensitive and inefficient, were not available in
%small packages, and were potentially dangerous in unsupervised applications
%such as sensor deployments.
%Additionally, the theoretically infinite lifetime of ceramic and tantalum
%capacitors is extremely attractive for sensor designers. It is possible to
%design sensors with power supplies with indefinite lifetimes, seemingly paving
%the way to pervasive and perpetual sensing.
Unfortunately, capacitors offer very low energy density, often much less than 1~Wh/L (compared to ~100~Wh/L for a small LiPo cell), and devices that use
them must cope with the limitations this presents. These include lower energy
utilization (if a device could harvest but the capacitor is full), reduced
availability (a sensor cannot perform a required task because its capacitor is already depleted), decreased responsiveness
(if the capacitor does not have energy to detect and respond to an event), and the inability
to easily perform high-energy tasks (if the capacitor does not have enough total
energy storage for task completion).

Many solutions have been proposed to tackle
these problems. Software checkpointing and associated programming models
that ease their use allow relatively high-energy tasks to be completed
correctly~\cite{luciaIntermittent17}. The introduction of wakeups based on the
sensed phenomena decreases the probability of missing an
event~\cite{campbellEnergy14}. Federating
energy stores for each predefined component workloads decreases latency between
tasks and eases modularity~\cite{hesterTragedy15}. Reconfigurable energy
buffer sizing lowers inter-task latency for arbitrary workloads~\cite{colinReconfigurable18}.
Even with these solutions, however, it is still not easy to program and
use capacitor-based energy harvesting sensors, and they still have decreased
availability and energy utilization compared to a solution with larger
energy storage capacity.


%This leads to an intermittent mode of operation where devices slowly
%work through a task, waiting until their capacitor bank is full before continuing


%intermittently, cycling between performing tasks until stored energy is
%depleted and off and recharging.
%Due to the small energy capacity of capacitor buffers,
%these systems are known as intermittent, as
%they are only as reliable as their source of energy, which is often
%unpredictable.

%This design has remained
%the status quo for cutting-edge energy harvesting sensor research for the last decade.
%before this sentence had problems in parrallelism -> federating = reconfiguring
%While recent designs have explored new techniques including federating the capacitor
%bank to better match the energy of a system component to its predefined task
%~\cite{hesterTragedy15}, or dynamically reconfiguring the size of a
%capacitor bank to minimize the latency between tasks in a given
%workload~\cite{colinReconfigurable18}, these designs
%still depend entirely on capacitors (or occassionally supercapacitors)
%suffer the costs associated with small energy
%capacity.

%For the past decade, researchers have tackled with the difficulties using and
%programming devices that depend on capacitor storage, also known as
%intermittent systems.
%These devices are known
%as intermittent systems because their low-capacity capacitor buffers are quickly
%depleted when performing tasks, sometimes in the middle of computation.
%In the worst case, these devices can get stuck in a sisyphean loop of turning
%on, attempting to perform a task, and running out of energy.  This phenomena
%makes these devices difficult to program, as their operation is tied to the
%volatility of immediately available harvestable energy. Additionally, the low energy capacity of a capacitor buffer

While capacitors may have been the best choice for energy harvesting sensors a decade
ago, this decision should be reevaluated in the face of new, modern technology.
We consider the changes in battery chemistries and associated
battery management ICs, and we argue that for modern sensor designs, a power
supply that offers an indefinite lifetime may not justify the cost of low energy
capacity. We believe that new battery technologies address many of the original
assumptions that limited their early adoption, and while they do not provide theoretically
infinite lifetimes, nor will they operate in all sensing scenarios, they may be
a better choice for the vast many common sensing tasks. It is time to reconsider the
use of batteries in ultra-low power energy harvesting sensors.

%they may provide lifetimes that are longer than the usable lifetime of most sensors.
%We believe it is time to reevaluate energy harvesting sensor s
%decreased
%usability, energy utilization, and availability, as well as the inability to
%perform long-running computations and tasks.
%Additionally, if one considers the
%finite lifetime of other components in a sensor, the usefulness of of an
%indefinite power supply is diminished.  After a decade of designing
%intermittent systems, we believe it is time to reevaluate energy harvesting
%sensor storage, and suggest that modern rechargeable batteries provide
%substantially more energy capacity, greatly improve upon the inadequacies of
%older batteries, and possess a finite, but sufficiently long lifetime.

%Batteries have been
%the preferred method of powering sensor nodes for both academic
%and commercial applications. They enable sensors that are easy to design, simple to program,
%and reliable to operate until their batteries die (and must be replaced).
%However, as we look toward dense and ubiquitous sensor deployments aimed
%at supporting applications such as building automation and industrial monitoring,
%the human costs of battery replacement are beginning to seem untenable. This
%is especially true if battery replacement must occur frequently.
%
%To address this issue, many researchers have explored the potential to power
%sensor systems through energy harvesting. Starting with outdoor solar
%powered nodes and eventually moving to nodes which scavenge power from the
%vibrations of our transportation infrastructure,
%ambient indoor lighting, or the thermal gradient
%between pipes, we have designed systems to monitor our world\textemdash at least
%when they have enough power. Most recently, a new class of
%computing systems has embraced this uncertainty by buffering small
%amounts of energy and intermittently working through
%a sensing task. These intermittent systems have eschewed both rechargeable
%and non-rechargeable batteries as expensive, failure-prone, fragile, and
%temperature sensitive, instead choosing to adopt capacitors as the
%more resilient option in their push
%toward perpetual sensing.
%
%While embracing capacitors increases the theoretical lifetime of such systems,
%lifetime gains come at the cost of usability and availability. Small energy
%stores lead to less utilization of the available ambient energy,
%have inherently lower availability, and experience higher degrees
%of intermittency that require
%complex programming models designed to micro-manage the remaining system energy.
%For many applications, especially those that are controlling spaces and
%systems which are human-facing, unpredictable degradation in availability is worse than
%the human costs of predictable battery replacement.
%
%In this work we show that by adopting batteries in energy
%harvesting systems rather than eschewing them, we can address these problems
%for the great majority of sensing workloads and harvesting scenarios.
%By modeling the charge/discharge patterns of energy buffers for periodic
%and event-driven workloads under real energy harvesting conditions, we find that rechargeable
%energy stores the size of a small battery achieve near maximum ambient
%energy utilization and significant availability gains over their capacitor-sized
%counterparts. To abolish the remaining bits of unavailability
%that are present during periods of high activity and long energy droughts, we propose the
%use of a backing energy store, most likely a non-rechargeable battery, and we
%show that the lifetime of this battery is improved by 4-6x over similar
%systems without energy harvesting. And while prior work is correct
%that these batteries will not operate at extreme temperatures, our analysis
%of the key complaints against batteries show that they are largely unfounded,
%outdated, or easily mitigated, and ultimately that batteries are well
%suited for powering sensor nodes in the most common sensing environments:
%those occupied and used by people.
%
%We implement these design decisions in a platform named \name and
%present the new and lower-power components selected for its design. The power measurements
%of these components are used as the basis for the standard workloads that
%we define and model.
%To verify function of the model, we deploy \name along with
%a battery-powered and intermittent
%system and compare their functionality to the functionality
%predicted by the model. We show
%that the timing of charge discharge cycles,
%and expected number of transmissions
%are similar to that of the deployed systems.
%
%By using batteries along with energy harvesting, we believe that we can
%create extremely long-lived and reliable sensor nodes.
%These sensor nodes will not need to micro-manage
%their energy state and can instead adapt their lifetime and energy
%usage over the course of weeks, months, or not at all. This push
%toward availability will enable more standard programming models and
%ultimately useful applications to be built on top of dense and ubiquitous
%sensor deployments.





%Some key thoughts on framing
%    - We need to make sure that it comes through this is targeting what
%    we define as the ``common'' use case.
%        - We should be clear about what we consider the common use case,
%        and I think it's obvious from our current direction that this should
%        be indoor/building sensing (Although we could consider trying
%        and outdoor scenario?? higher harvesting potential better
%        drives the point of having a rechargeable battery.)
%    - When we talk about availability it should be clear how we are defining
%    it and why it is important. Really it is the important metric (along with
%    lifetime). I'm going to try to look for some citations about
%    how humans respond to failure. My sense has always been that even low
%    failure rates are really frustrating, but we should have some data to back
%    this up.
%    - availability also means liveness. This is the most frustrating thing
%    about deploying intermittent sensors! is it dead or just not getting
%    energy? It's hard to pre-emptively catch failure if you can't test
%    failure until it's too late.
%    - There is this idea that we designing a more usable node by using
%    batteries. Rather than micro-optimizing every transaction batteries
%    allow us to macro-optimize energy and applications over time. This could
%    mean adapting your application without missing your deadline (i.e occupancy
%    sensors that only sense lack of occupancy every 10 minutes instead of every
%    minute).  It's unclear to me how much this should come through. I definitely
%    think that intermittency pushes functionality away from the edge, and
%    reliablity pushes it toward the edge, but that is its own argument so we
%    probably have to be careful.
%    - How much of a role does permamote play into this story. Currently it is relegated
%    to the platform that came out of these design decisions. This is not a platform
%    paper. It could arguably be a platform paper. What is the next version of
%    the sensor node everyone uses? Maybe it is the sensor that integrates
%    this work on energy harvesting. The current paper is not framed this way though
%    so we should decide sooner rather than later if that's a good idea. I generally
%    think that platform papers are more hit or miss.
%        - If permamote plays a role we should absolutely include its components
%        selection matrices. Not the trendlines, but the lines motivating that
%        these really are the best parts and by choosing them we get significant (2-4x)
%        lifetime improvements over designs that are even a few years old. (AKA stop using the RF233).
%When someone is done reading this paper what do we want them to take away?
%    - Using batteries primary and secondary cells
%    on sensor nodes prioritizes the common case
%    (long lifetime, normal environment, high availability) which we are currently leaving behind.
%    - For very little physical volume we can completely avoid intermittency for a very
%    long time and that is a worthy tradeoff. Even if your primary cell dies
%    you are no worse than an intermittent node.
%    - Batteries are not as bad as everyone thinks and they are getting much
%    better in time.
%    - Something about node size? This was toyed at in the grafting paper, and
%    I think we have the chance to really nail it. At what point can do the following tradeoffs occur:

