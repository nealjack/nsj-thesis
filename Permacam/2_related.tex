\section{Related Work}
The development of smart wireless camera platforms is a natural follow up to original sensor network research from the 2000's. The advent of efficient commercially available CMOS image sensors led to emergence of many wireless camera platforms. These platforms largely shared a goal in enabling interesting indoor computer vision applications while minimizing energy usage.

Many of these platforms focused on performing local image processing instead of transmitting images.
The Cyclops~\cite{rahimi2005cyclops} platform was built with the goal of performing all image processing locally on-device. It consisted of an integrated camera with external memory and processing, and was connected to a MicaZ mote for networking. The designers chose to use a low-power 8-bit processor for the camera core, limiting the platform's capability to some simple movement detection and hand gesture recognition.
CITRIC \cite{chen2008citric} follows a similar integrated hardware design to Cyclops, but chose a significantly more powerful image processor. This design decision prioritizes the ability to perform image processing like target tracking and camera localization. CITRIC's powerful processor requires an order of magnitude more active and idle power compared to Cyclops, however. The designers expect the platform to last just 16 hours on 4 AA batteries. 

%, SensEye \cite{kulkarni2005senseye}, FireFly Mosaic\cite{rowe2007firefly}, and CITRIC \cite{chen2008citric} platforms focused on developing systems that could capture still images and frame sequences, transmit them, and perform lightweight onboard image processing. CITRIC and Cyclops demonstrated object detection and tracking, image localization, hand gesture recognition. Firefly Mosaic was built on top of the CMUcam3, which offered numerous builtin utilities like edge detection and face detection. A distributed network of Firefly Mosaic motes was used for detecting various activities within a home.

Other wireless platforms emerged that focused on intelligently transmitting frames.
FireFly Mosaic \cite{rowe2007firefly} was built for human activity recognition and tracking at high temporal granularity. Due to its application requirements, the platform requires tight time synchronization, and high frame rates. These real-time requirements present a significant burden on the platform and result in only a 5 day lifetime on 4 AA batteries.
%The Panoptes~\cite{feng2005panoptes} platform was built to explore minimizing energy by only transmitting interesting video frames. The platform used a USB camera, and sent frames over an 802.11 link. The choice to use non-integrated and unoptimized components like a USB camera, continuously capture video, and use WiFi increases power dramatically. 
The SensEye~\cite{kulkarni2005senseye} system utilized a network of hierarchical camera motes to reduce energy usage over the entire network. Less capable and lower power motes were used as wake up sources for more powerful platforms. Similarly, MeshEye used two low power, low resolution cameras for stereoscopic distance measurement and as a wake up for a more capable camera.

While all of these platforms were limited by available technology, many could have made different design decisions to significantly reduce power and increase potential lifetime. Platforms like CITRIC and Firefly Mosaic were overly focused on performing powerful local processing or producing high frame rates that they were no longer low power. SensEye and MeshEye omit simple and more efficient methods for low power wake up. Both Cyclops and Firefly Mosaic mention the utility of using a low power wake up like a PIR sensor to reduce idle energy usage and unnecessary image captures, but do not implement it. 

%Another camera platform, SensEye \cite{kulkarni2005senseye}, uses simple low power wireless cameras as wake up mechanisms for higher power, more capable camera motes. The authors note that this low power wake up mechanism conserves significant energy system-wide. Similarily, MeshEye \cite{hengstler2007mesheye} used two low power 30x30 stereo pixel arrays to detect moving objects and wake up a more powerful color camera.

%SensEye~\cite{kulkarni2005senseye}, MeshEye~\cite{hengstler2007mesheye}, MicrelEye~\cite{kerhet2007low}, and Panoptes~\cite{feng2005panoptes} were developed to perform limited video streaming and onboard object classification, detection, and tracking. Many of these platforms also explored novel hierarchical network configurations and dynamic image resolution scaling to adapt to lifetime goals.

%All these platforms are limited by the technology available when they were created. They are not low power enough to support a sufficiently long lifetime, and that has limited their adoption. These platforms require on average hundreds of milliwatts of power, leading to lifetimes of a few days \cite{rowe2007firefly} of streaming images to a couple months at most \cite{hengstler2007mesheye} transmitting image summaries. This problem is inherently tied to the technology available when these platforms were created. In the past decade, improvements in MCUs, Radios, and image sensors has led to orders of magnitude reduction in power \hl{need citations?}.

Modern platforms have taken advantage of technology improvements, as well as employed new techniques to reduce transmit power and increase (or abolish) lifetime. WISPCam~\cite{naderiparizi2015wispcam}, is a battery-free camera that utilizes RFID for power and backscatter communication. It can capture an image every 15 minutes when an RFID reader is 5 meters away. BackCam~\cite{josephson2019wireless} is  another camera platform that utilizes backscatter for communication, but over commodity WiFi. BackCam can stream video at 4 frames per second for a lifetime of 32 days. Similar to WISPCam, BackCam also requires a nearby wall-powered transmitter to generate excitation packets for backscatter communication. Camaroptera~\cite{nardello2019camaroptera} is another batteryless platform, focused on wide-area outdoor applications. The platform uses a long-range LoRa radio and harvests energy from solar panels. It performs local inference to detect people within captured photos and can achieve an end-to-end latency of less than 20 seconds in well-lit outdoor environments. Commercial platforms like the Blink Indoor wireless security camera~\cite{blinkindoor} and the Wyze outdoor camera~\cite{wyzeoutdoor} utilize low power motion detection to minimize energy usage. The Blink Indoor claims a 2 year lifetime consisting of 40000 seconds of recording 720p video on two AA batteries. Wyze estimates a lifetime of three to six months on its internal 5.2\,Ah battery if capturing 10 to 20 video clips per day.

While these modern platforms push the envelop for video streaming or batteryless imaging, they have drawbacks that limit their deployability in indoor spaces. Backscatter-based systems require a nearby transmitter to communicate. Even with a dedicated transmitter, WISPCam can only periodically capture images every 15 minutes. Camaroptera is designed for outdoor use, and it is unclear how frequent it can capture and send images indoors, especially as its energy harvesting system requires 197\,\uW when idle. The Blink camera lasts two years if it is placed in locations infrequently occupied by people. Placing the camera in a kitchen that sees an average 26\% occupancy on weekdays leads to a lifetime of only 1.78 days \cite{josephson2019wireless}. The Wyze camera faces a similar fate if placed in a frequently occupied area.
We seek to fulfill the indoor image sensor design point with \name{}, a platform that supports a multi-year indoor lifetime with minimal infrastructure.