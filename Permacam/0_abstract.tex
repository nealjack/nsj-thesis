Deploying widespread computer vision applications indoors is traditionally burdened by costly installation and maintenance. Cameras are generally power intensive and thus necessitate wired connections. This limits placement options, coverage, and potential applications. Wireless, battery-powered cameras exist, but generally do not last longer than a few weeks or months without battery replacement or recharging. This severely hinders the permanence and usefulness of indoor computer vision applications.

In this paper, we present \name, a wireless camera sensor platform with a multi-year lifetime. \name combines energy harvesting with a traditional battery to last  half a decade when configured to capture images every ten minutes in an indoor environment. The platform is capable of performing local image inference like person classification and transmitting images end-to-end. We perform an in-depth exploration into the tradeoff between processing locally compared to relaying and processing elsewhere. 

From our findings, we conclude that in most cases local computation does not provide enough benefit in lifetime or capability to outweigh the usability and flexibility of sending images to a more capable endpoint. Thus, we develop an end-to-end image transfer architecture that integrates easily with powerful and easy to use computer vision frameworks. This system enables researchers to quickly deploy wide-reaching and long-lasting computer vision tasks like object detection and tracking. By dramatically lowering installation and maintenance costs, \name could lead to increased adoption of existing computer vision applications for building energy efficiency, smart cities, and beyond.


