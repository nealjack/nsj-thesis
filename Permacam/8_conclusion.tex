\section{Conclusions}

In this work, we describe \name, the first indoor, wireless camera sensor with a multi-year lifetime. This lifetime is achieved through energy harvesting and a hierarchical energy storage architecture. 
\name is capable of performing simple inference like image classification locally, as well as end-to-end image transmission to endpoints capable of complex computer vision tasks like object detection. \name supports a lifetime of half a decade while capturing an image every ten minutes.

We examine the trade off between both local computation and transmission within the scope of indoor sensing. We discover that currently, image transmission is more energy efficient and allows more capable and accurate inference when compared to local computation. However, this conclusion is in flux as new technology to accelerate inference at the edge is on the horizon. We design \name as a platform that can readily integrate new processors and accelerators for local inference.

In the meantime, \name is capable of supporting flexible and powerful computer vision applications now with end-to-end image transmission. We envision deployments of \names driving applications that monitor critical remote infrastructure, provide accurate and high fidelity occupancy counts to lower building energy usage, and provide ground truth for contact tracing efforts.

